\section*{\centering \underline{DEDICATORIA}}

Facundo.- Le dedico este Trabajo Final de Grado a mi familia; a mi madre Analía que me acompañó fisica y emocionalmente durante toda la carrera y hoy lo hace desde el cielo , a mi padre Eligio y a mis hermanos Luciano y Benjamín que me apoyaron en todo momento. 
También quiero dedicarles este Trabajo a mis amigos y compañeros de la universidad que estuvieron a mi lado incondicionalmente.

A dios, por sobre todas las cosas por haberme guiado y acompañado en este camino.
\newpage

\section*{\centering \underline{AGRADECIMIENTOS}}
COMPLETAR \\
\newpage
\section*{\underline{TÍTULO DEL PROYECTO}}
Diseño, desarrollo e implementación de un sistema de gestión de indicadores relacionados a programas de formación.\\
\section*{\underline{HOJA DE ACEPTACIÓN DEL TRABAJO FINAL}}
COMPLETAR \\
\section*{\underline{ÍNDICE}}
COMPLETAR INDICE

\section*{\underline{GLOSARIO Y LISTADO DE SÍMBOLOS Y CONVENCIONES}}
COMPLETAR \\
\section*{\underline{RESUMEN}}
COMPLETAR \\
\section*{\underline{PALABRAS CLAVES}}
COMPLETAR

\newpage

\chapter{\centering \underline{Introducción}}

\section{\underline{INTRODUCCIÓN}}
COMPLETAR INTRODUCCIÓN.\\

\section{\underline{OBJETIVO DEL PROYECTO}}

Objetivo general:\\

Contribuir, posibilitar y facilitar el seguimiento de los programas de formación e inclusión tecnológica para mujeres adolescentes mediante el desarrollo e implementación de un sistema de seguimiento, control y obtención de métricas para estos programas.
A raíz de estos indicadores se busca encontrar señales y problemáticas a resolver que ayuden a disminuir la brecha de género en el área de la tecnología en la región.\\

Objetivos específicos:
\begin{itemize}
	\item Proporcionar una herramienta que según los datos almacenados en ella permita obtener indicadores de:
	\begin{itemize}
	\item Qué porcentaje de mujeres adolescentes completan los cursos y luego inician una carrera relacionada a la tecnología.
	\item Qué porcentaje de mujeres adolescentes abandonaron los cursos, lugar y motivos.
	\item Qué porcentaje de mujeres adolescentes viven con familiares y cómo es su parentesco.
	\item Qué cursos poseen más inscriptos y cuales más desertores.
	\item Situación socio-económica de las mujeres adolescentes que realizan los cursos.
	\item Preferencias de carreras universitarias o terciarias de estas mujeres
	\item Qué porcentaje de mujeres trabajan y estudian al mismo tiempo.
	\end{itemize}
	\item Identificar y permitir la visualización de indicadores relevantes y representativos que ayuden al seguimiento de los programas.
	\item Proporcionar un sistema de notificaciones que comunique a los distintos tipos de usuarios sobre eventos que puedan ser previamente configurados por ellos.
	\item Permitir la configuración de distintos perfiles con diferentes niveles de privilegios para utilizar el sistema.\\
\end{itemize}

\section{\underline{DESTINATARIOS}}
El Trabajo Final de Grado se realizó en el marco del proyecto UNDEX 775/2019 aprobado al Depto. de Computación e Informática de la Facultad de Ingeniería del CRUC IUA (Resolución 441/2019), en la convocatoria 2019 de UNDEF \textbf{\cite{ResolucionUndex}}. En dicho proyecto se desarrollará la estrategia a implementar mediante un sistema software para gestionar el seguimiento de los programas de capacitación tecnológica para favorecer la inclusión de la mujer. \\


\section{\underline{BENEFICIOS}}


\chapter{\centering \underline{Marco Teórico}}
\section{\underline{TÍTULOS A DESARROLLAR}}
\section{\underline{TÍTULOS A DESARROLLAR}}
\section{\underline{TÍTULOS A DESARROLLAR}}

\chapter{\centering \underline{Desarrollo}}

\section{\underline{ESTUDIO TÉCNICO}}

\section{\underline{DESARROLLO DEL TRABAJO}}

\section{\underline{INVERSIÓN REQUERIDA}}
Para el desarrollo de este Trabajo Final de Grado no se requirió ninguna inversión.


\section{\underline{PROYECCIÓN DE COSTOS DE OPERACIÓN Y MANTENIMIENTO}}
El sistema se desplegó en un servidor físico por lo cual los costos de operación y mantenimiento fueron los relacionados a los gastos de luz e internet que han sido a cargo del CRUC IUA. \\

\section{\underline{ANÁLISIS DE VIABILIDAD COMERCIAL}}
No corresponde dado que se enmarcó en un proyecto UNDEX.\\

\section{\underline{ANÁLISIS FINANCIERO}}
No corresponde dado que se trató de un trabajo final de carrera.\\


\section{\underline{ESTUDIO AMBIENTAL}}
La elección de un servidor local o en la nube trae impacto ambiental negativo.\\

Mantener toda la información en la nube da la posibilidad de no tener que preocuparse por perder y/o procesar información ya que los proveedores de servicios en la nube permiten hacerlo desde cualquier lugar, pero esto significa tener la información en datacenters compuestos de miles de servidores que utilizan una cantidad de energía inmensa. Para mantener estos edificios se necesitan kilómetros de fibra óptica e infraestructura que requieren energía en el camino. En el centro, los datos se almacenan varias veces en discos duros, y la actividad constante de todos esos discos genera mucho calor, lo que requiere aires acondicionados que consumen mucha energía para proteger el equipo del sobrecalentamiento.\\
Además, los proveedores de servicios en la nube generan toneladas de desechos electrónicos, aunque algunos de estos proveedores, como Microsoft, están planeando proyectos de reciclaje y redistribución de los servidores y componentes degradados o en desuso.
\\

Mantener un servidor local, permite controlar de manera más eficiente el uso de la energía y de los componentes, pero no se tiene la ventaja que provee la nube en el procesamiento y almacenamiento de los datos.\\

\section{\underline{ESTUDIO SOCIAL}}
La industria tecnológica y de desarrollo de software ha tenido un gran crecimiento en los últimos años. La Argentina no es ajena a este proceso: Actualmente la industria del software posee alrededor de 115.000 empleos y se proyecta para el año 2030 unos 500.000 en total o incluso más \textbf{\cite{EmpleosUltAnios}}(Matias K, 2021).\\

En una industria en pleno crecimiento, la inserción de la mujer sigue presentando desafíos:
\begin{itemize}
	\item Aunque en los últimos 15 años se duplicó la participación de mujeres en la industria de IT, todavía la brecha frente a la participación de hombres en esta misma industria es muy grande (70\% hombres, 30\% mujeres).  \textbf{\cite{DuplMujeres}}(Mujeres en la industria del software, 2020)
	\item Solo el 30\% de los estudiantes de carreras de Ciencia, Tecnología, Ingeniería y Matemática (CTIM) que se registran en universidades públicas y privadas son mujeres. \textbf{\cite{MujeresCtim}}(Ana inés \& Cecilia, 2019)\\
\end{itemize}

En adición, la industria de Software es una de las pocas que ha mantenido su crecimiento durante los últimos años y posee uno de los mejores salarios en Argentina.\\

La brecha de género afecta también a la innovación y al crecimiento tanto del campo tecnológico como el desarrollo económico y social de un país. Los programas de inclusión vienen a tratar de disminuir esa brecha de género en tecnología, acercando experiencias formativas a mujeres adolescentes.\\

Es de suma importancia poder tener la posibilidad de realizar un seguimiento de cada alumno que hace estos programas de inclusión para poder sacar conclusiones, métricas y tomar decisiones con el fin de mejorar estas experiencias y por qué no, aumentar la cantidad de mujeres en el rubro de la tecnología. \\

\section{\underline{EVALUACIÓN ECONÓMICA}}
No corresponde dado que se trató de un trabajo final de carrera.

\chapter{\centering \underline{Conclusión}}